\documentclass{article}
\usepackage{verbatim}
\usepackage{longtable}
\usepackage[margin=1.0in]{geometry}

\title{VLSEE: Very Large Scale Electrophysiology-Experimental}
\author{}
\date{}
\begin{document}
\maketitle

\tableofcontents

\section{Introduction}
VLSEE is a collection of scripts used in and with a large scale spike sorting
process. A section is dedicated to each of the scripts detailing behavior
and usage. [Deprecated] proceeds the titles of scripts that are no longer
actively maintained or improved.

\subsection{Note on \texttt{custom/}}
Scripts in the \texttt{custom/} directory, such as
\texttt{plot\_penultimate}, are modified versions of original core
subroutines with either performance enhancements or changes needed to
better run scripts in VLSEE.

\subsection{Note on parameters}
Many routines have additional parameters not passed to them as
arguments. This is in an attempt to simplify operation for users--though
``power'' users are free to edit ``hidden'' parameters defined within
functions. Note that the values of these parameters are constantly
subject to refinement as the functions are further tested with newer
datasets. 

\section{\texttt{simple\_snr}}
\texttt{simple\_snr} is a function that calculates the mean signal-to-noise
ratio (SNR) and simply returns an array of units whose mean SNR falls below
a given threshold.

\subsection{Usage}
Example:

\texttt{badunits = simple\_snr(dounits, bestchannel, wavedir, TL)}
\begin{center}
\begin{tabular}{l r}
Parameters&Meaning\\
\texttt{dounits}&Vector of indices of units to do\\
\texttt{bestchannel}&Vector of best channel for each unit\\
\texttt{wavedir}&Path to waveforms files\\
\texttt{TL}&Low SNR threshold (not in dB)

\end{tabular}
\end{center}

\section{\texttt{get\_sane}}
\texttt{get\_sane} is function that returns an array of indices of bad units
based on some simple criteria. Note that bad units are considered to be those
that fall \emph{outside} the valid range of these parameters. It should function
as a basic sanity check before any sorting process to prevent garbage units
from causing a ``garbage--in--garbage--out'' effect.
\begin{center}
\begin{longtable}{l c r}
Value&Minimum&Maximum\\
Peak-to-peak Voltage (V\ensuremath{_{pp}})&\ensuremath{60 \mu}V&\ensuremath{350 \mu}V\\
Absolute Voltage&\ensuremath{-350 \mu}V&\ensuremath{350 \mu}V\\
Wave Count&200\footnote{This minimum is 0 if this function is run at the penultimate
step}&\ensuremath{\infty}\\ 
Coefficient of Variation (Magnitude)&0&0.25
\end{longtable}
\end{center}

\subsubsection{Note on Coefficient of Variation}
The calculation of this parameter is motivated by the observation that most
units that are considered to be artifacts are very noisy around the peaks.
Here, the mean of the lowest point of the unit is considered to be the peak
and the coefficient of variation is defined as:
\begin{equation}
        c_v = \sigma/\mu
\end{equation}
were \ensuremath{\sigma} is the standard deviation of the minimum of the
waveforms and \ensuremath{\mu} is the minimum of the mean waveform. The
absolute value of the coefficient of variation is used in the function to
simplify the process.

\subsection{Usage}
Example: 

\texttt{badunits = get\_sane(do\_units, spiketimes, bestchannel, wavedir, sampling\_rate, ispen)}
\begin{center}
\begin{longtable}{l r}
Paramters&Meaning\\
\texttt{do\_units}&The indices of units to check\\
\texttt{spiketimes}&The cell array of spike times\\
\texttt{wavedir}&The director where the waveforms are store\\
\texttt{sampling\_rate}&The sampling frequency\footnote{currently unused}\\
\texttt{ispen} (1=True, 0=False)&Whether this function is being run just after \texttt{get\_penultimate\_units}
\end{longtable}
\end{center}

\section{\texttt{pca\_merge}}
\texttt{pca\_merge} is a script that attempts to use MATLAB's Principal
Component Analysis (\texttt{pca}) tool to guide an algorithm in merging units. 
Two units are first compared by concatenating their waveform data and
taking the z-scores of this data (to compensate for the fact that
\texttt{pca} is not a scale-invariant process). Note that each column of
the concatenated waveforms represents values of a given variable. The
number of variables is determined by the number of sample points per
unit. \texttt{pca} is then used to transform the data such that 
the first few components or variables contain most of the information of
the changes in the data. It is expected then, for two distinguishable
units that plotting the first two components of the transformed data
will yield two clusters in 2-D space. Currently \texttt{pca\_merge}
considers the first three dimensions of the transformed data and takes
the distance between the centers of the two clusters as the basis for
merge decisions. 

Additionally, \texttt{pca\_merge} performs the additional basic sanity
checks of only comparing units that share the same best channel and
units that have max voltages and min voltages within a certain
tolerance. Both of these additional sanity checks are pending
removal--to simplify the process.


\subsubsection{Note on Z-score}
Given a value of a variable \ensuremath{x}, the z-score of this value is
defined as:
\begin{equation}
z =\frac{ x - \mu}{\sigma}
\end{equation}
where \ensuremath{\mu} is the mean of \ensuremath{x} and
\ensuremath{\sigma} is the standard deviation of \ensuremath{x}.

\subsection{Usage}

Example:
\texttt{samemeans = pca\_merge(dounits,bestchannel,wavedir)}
\begin{center}
\begin{tabular}{l r}
Parameter&Meaning\\
dounits&Vector of indices of units to do\\
bestchannel&Vector of of best channels of units\\
wavedir &Path to waveforms
\end{tabular}
\end{center}

\section{\texttt{custom/plot\_penultimate}}
This version of \texttt{plot\_penultimate} contains several changes to
the version in ``core subroutines.'' \texttt{plot\_penultimate} has
currently been unused in recent iterations of VLSE, but it remains
important in understanding problems during the
\texttt{get\_final\_units} process. One of the reasons that the old
version of this script was unused was that it takes an excessively long
amount of time to plot the number of units (on the order of
\ensuremath{10^3} typically present after the
\texttt{get\_penultimate\_units} process. 

\section{[Deprecated] \texttt{density\_distance\_area\_merge}}
\texttt{density\_disance\_area\_merge.m} was an experimental script
that attempted to establish a criteria from the merging of two units
based on the similarity of of the distribution of their interspike
intervals (ISI)s. 
Unfortunately, while this idea works well in theory, units that are
``incomplete'' or ``immature,'' meaning that they do not have most of
the waveforms supposed to be associated with them actually included,
tend to have ISIs that deviate significantly from the ideal case. Or
equivalently, that a unit is incomplete often has a significant impact
on its ISI distribution, meaning that this idea works well only with
units that have enough spike times such that the distribution is
well-defined and will not change significantly with the addition of more
spikes from a unit it should be merged with. To compare actual
distributions, this script separated the ISI data into bins of
predetermined size and performed a discrete summation of absolute value
of the difference of the distributions of two units and compared the
result to an empirically defined parameter.

\section{[Deprecated] \texttt{quasi\_pdf\_merge}}
\texttt{quasi\_pdf\_merge} was an experimental script that attempted to
also use the distribution of the ISIs of two units as a merge criteria,
but in addition to the problems encountered with
\texttt{density\_distance\_area\_merge}, \texttt{quasi\_pdf\_merge}
encounters additional problems. Namely, as this script relies on first
fitting a distribution to the ISIs of a given unit, the validity of the
comparison of the distribution is dependent not only on the number of
spike times of the unit, but the ``goodness of fit.'' In practice, the
``goodness of fit'' was usually not good enough to justify use of this
function. Additionally, even with a ``good fit,'' it remains troublesome
to identify a good metric for comparing different probability density
functions. The approach of simply looking at the different in the
parameters of some probability density function was used. Note that if
two discrete probability distributions are to be compared, the approach
of \texttt{density\_distance\_area\_merge} is probably better.

\section{[Deprecated] \texttt{visualize\_quasi\_pdf}}
\texttt{visualize\_quasi\_pdf} is a script that attempts to fit various
probability distributions to the ISI distribution. Distributions are
plotted with a histogram of the ISI.

\section{[Deprecated] \texttt{principal\_component\_example}}
\texttt{principal\_component\_example} is a demo program that creates
visualizations of the process used in \texttt{pca\_merge}. It plots the
first two components of the transformed data, allowing for the
separation (or lack thereof) of clusters of waveforms to be seen.

\end{document}
